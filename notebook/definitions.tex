\subsection{Définitions, acronymes et abréviations}

\begin{description}

\item[UI]
Acronyme de “user interface” (interface utilisateur).

\item[Corpus]
Un corpus est un ensemble de documents, artistiques ou non ( textes, images,
vidéos, etc. ) , regroupés dans une optique précise. 

\item[Fichier]
 Contenant virtuel auquel est assigné un nom unique, permettant de classifier et
de réunir en une même entité une séquence de données. Le fichier est stocké dans
un système de fichier et les données qu'il contient sont généralement
structurées en suivant un même format.

\item[Document]
En informatique, le mot \textit{document} est généralement synonyme de fichier.
On
parle ici de document électronique. Un document électronique est un contenu de
médias électroniques ( autres que les programmes d'ordin\\-ateur ou des fichiers
système ) qui sont destinés à être utilisés soit dans une forme électronique ou
comme sortie imprimée.

\item[Indexation]
L'indexation permet de regrouper en un seul endroit toutes les données
souhaitées. On crée des indexes, ce qui permet d'y accéder plus rapidement. 

\item[Index]
Un index est, en toute généralité, une liste de descripteurs à chacun desquels
est associée une liste des documents et/ou parties de documents auxquels ce
descripteur renvoie. Lors de la recherche d'information d'un usager, le système
accèdera à l'index pour établir une liste de réponses. 

\item[Moteur de recherche]
Un moteur de recherche est un code logiciel qui est conçu pour rechercher des
informations ou retrouver des ressources associées à des mots quelconques. Ici,
on parle de moteur de recherche de type “ Desktop ” , car son champ d'action est
limité à l'ordinateur sur lequel l'application est installée. 

\item[Requête]
 En informatique, une requête est une demande de traitement. Dans notre cas, le
terme est employé dans le contexte des bases de données, une requête
correspondant à l'interrogation d'une base pour en récupérer une certaine partie
des données.
 
\item[Base de données]
 Une base de données, usuellement abrégée en BD ou BDD, est un ensemble
structuré et organisé, permettant le stockage de grandes quantités de
d'informations afin d'en faciliter l'exploration ( ajout, mise à jour, recherche
de données ). Autrement dit, il s’agit d’un conteneur informatique permettant de
stocker dans un même endroit l'intégralité des informations en rapport avec une
activité. Une base de données permet de stocker un ensemble d'informations de
plusieurs natures ainsi que les liens qu'il existe entre les différentes
natures.

\item[Expression régulière]
 Les expressions régulières ( aussi appellées expressions rationnelles ) sont de
chaines de caractères permettant de décrire un ensemble de variables par
l'utilisation d'une syntaxe précise qui se retrouvent dans de nombreux langages
et outils.
 
\item[Démon]
 Un démon ou daemon désigne un type de programme informatique, un processus ou
un ensemble de processus qui s'exécute en arrière-plan plutôt que sous le
contrôle direct d'un utilisateur.

\end{description}
