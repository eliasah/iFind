\section{Annexe}

\subsection{Communication du moteur d'indexation vers la base d'indexation (DTD)}
\begin{verbatim}
<!-- règle génerale :
chaque fichier Xml doit contenir la ligne suivante
<?xml version="1.0" encoding="UTF-8"?> -->

<!-- comme convenue toutes les balises de l'indexation sont facultatives -->
<!ELEMENT INDEXATION (RENOMMAGES?, MODIFICATIONS?, SUPPRESSIONS?, CREATIONS?)>

<!ELEMENT RENOMMAGES (FICHIERRENOMME)*>		
<!ELEMENT MODIFICATIONS (FICHIERMODIFIE)*>	
<!ELEMENT SUPPRESSIONS (FICHIERSUPPRIME)*>
<!ELEMENT CREATIONS (FICHIERCREE)*>		

<!ELEMENT FICHIERRENOMME (PATH, NEWPATH)>
<!-- un fichier renommé n'est pas nécessairement un fichier modifié -->
<!-- un fichier modifé nécessite une re-indexation --> 
<!ELEMENT FICHIERMODIFIE (PATH, DATEMODIFICATION, TAILLE, PROPRIETAIRE, GROUPE, 
PERMISSIONS, INDEXAGE, NEWPATH?)>
<!ELEMENT FICHIERSUPPRIME (PATH)>
<!ELEMENT FICHIERCREE (PATH, format, DATECREATION, TAILLE, PROPRIETAIRE, GROUPE, 
PERMISSIONS, INDEXAGE)>

<!-- les meta-données --> 
<!ELEMENT PATH (#PCDATA)>
<!ELEMENT format (#PCDATA)>
<!ELEMENT DATECREATION (#PCDATA)>
<!ELEMENT DATEMODIFICATION (#PCDATA)>
<!ELEMENT TAILLE (#PCDATA)>
<!ELEMENT PROPRIETAIRE (#PCDATA)>
<!ELEMENT GROUPE (#PCDATA)>
<!ELEMENT PERMISSIONS (#PCDATA)>
<!-- liste des mots a indexer -->
<!ELEMENT INDEXAGE (MOT*)>
<!ELEMENT MOT (#PCDATA)><!ATTLIST MOT frequence CDATA #REQUIRED>
<!ELEMENT NEWPATH (#PCDATA)>

<!-- les id's seront utlisés pour la traçabilitée et la détection d'eventuel 
erreurs, elle sont tout de même facultatives --> 
<!ATTLIST RENOMMAGES id CDATA #IMPLIED>
<!ATTLIST MODIFICATIONS id CDATA #IMPLIED>
<!ATTLIST SUPPRESSIONS id CDATA #IMPLIED>
<!ATTLIST CREATIONS id CDATA #IMPLIED>
\end{verbatim}

\subsection{Communication du moteur de recherche vers le moteur d'indexation (DTD)}
\begin{verbatim}
 <!-- Description de la requête de recherche -->
<!ELEMENT SEARCH (WORD, CONTENT?, PATHDIR?, PERM?, EXTENSION?, TIMESLOT?)>
<!-- Le mot à rechercher -->
<!ELEMENT WORD (#PCDATA)>
<!-- Un booléen qui dit si l'on fait une recherche de contenu (true) ou une 
recherche sur les nom de fichier (false) -->
<!ELEMENT CONTENT (#PCDATA)>
<!-- Le nom du fichier à partir duquel on recherche -->
<!ELEMENT PATHDIR (#PCDATA)>
<!-- Les permissions du fichier à chercher -->
<!ELEMENT PERM (#PCDATA)>
<!-- L'extension des fichiers à chercher -->
<!ELEMENT EXTENSION (#PCDATA)>
<!-- Interval de temps -->
<!ELEMENT TIMESLOT (BEGIN, END)>
<!-- Debut de l'interval -->
<!ELEMENT BEGIN (DAY, MONTH, YEAR)>
<!-- Fin de l'interval -->
<!ELEMENT END (DAY, MONTH, YEAR)>
<!-- Le jour -->
<!ELEMENT DAY (#PCDATA)>
<!-- Le mois -->
<!ELEMENT MONTH (#PCDATA)>
<!-- L'année -->
<!ELEMENT YEAR (#PCDATA)>

<!-- L'id du resultat correspondant à une recherche aura le même id que la 
recherche -->
<!ATTLIST SEARCH id CDATA #REQUIRED>
\end{verbatim}

\subsection{Communication de la base d'indexation vers le moteur de recherche (DTD)}
\begin{verbatim}
 <!-- Balise contenant les résultats de la requête search -->
<!ELEMENT RESULT (FILE*)>
<!-- Balise file correspond à un fichier resultat donc n balises files = n 
résultats -->
<!ELEMENT FILE (NAME, PATH, PERM, SIZE, LASTMODIF?, PROPRIO?)>
<!-- Le nom du fichier -->
<!ELEMENT NAME (#PCDATA)>
<!-- Le chemin complet du fichier -->
<!ELEMENT PATH (#PCDATA)>
<!-- Les permissions du fichier -->
<!ELEMENT PERM (#PCDATA)>
<!-- La taille du fichier -->
<!ELEMENT SIZE (#PCDATA)>
<!-- La date de dérnière modification du fichier -->
<!ELEMENT LASTMODIF (#PCDATA)>
<!-- Le propriétaire du fichier -->
<!ELEMENT PROPRIO (#PCDATA)>

<!-- L'id du resultat correspondant à une recherche aura le même id que la 
recherche -->
<!ATTLIST RESULT id CDATA #REQUIRED>
\end{verbatim}
