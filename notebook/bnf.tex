\paragraph{BNF}
Nous avons défini une syntaxe et à fortiori, une grammaire sous forme BNF.

Grammaire (BNF) :\\
% \begin{tabular}{r c l}
% query &$\rightarrow$ & word { -or word } [options]\\
% query &$\rightarrow$ & word { word } [options] \\
% 	 
% options&$\rightarrow$ &-e word {word}\\
% word   &$\rightarrow$ &char*\\
% char   &$\rightarrow$ & [a-z A-Z 0-9 à é è ë ä ù ï ç]\\ 
% \end{tabular}

\begin{tabular}{p{1.5cm} p{0.5cm} p{9cm} }
& & \\
\textbf{char} & $\equiv$ & [:alnum:]\\
\textbf{word} & $\equiv$ & $\textbf{char}^+$\\
\textit{query} & ::= & \textbf{word} \textit{\{} \textbf{-or} \textbf{word} \textit{\}} \textit{[} \textit{options} \textit{]}\\
& | & \textbf{word} \textit{\{} \textbf{word} \textit{\}} \textit{[} \textit{options} \textit{]}\\
\textit{options} & ::= & \textbf{-e} \textbf{word} \textit{\{} \textbf{word} \textit{\}}\\
\end{tabular}

Par la suite, nous avons défini plusieurs types abstraits de données permettant de modéliser la grammaire.
Un type a forcément une ou plusieurs opérations de construction. 
Chaque type peut présenter des fonctions d'accès et de test ainsi que des constantes.

\paragraph{Type de données abstraites}
En effet, nous avons définit 6 types tels que :

\begin{tabular}{r l}
%\textbf{Type : operateur} & \\
%constantes :& plus, OR et Exclu : operateur
% note (isabelle) La présence des types [conj] et [disj] rend la constante [operateur] inutile.

\textbf{Type : requête} & \\
_construction :& cons-requete-word : word $\rightarrow$ requête\\
& cons-requete-word-options : word * options $\rightarrow$ requête\\ 
& cons-requete-conj : conj $\rightarrow$ requête\\
& cons-requete-conj-options : conj * options $\rightarrow$ requête\\
& cons-requete-disj : disj $\rightarrow$ requête\\
& cons-requete-disj-options : disj * options $\rightarrow$ requête\\
_test :& est-word : requête $\rightarrow$ bool\\
& est-conj : requête $\rightarrow$ bool\\
& est-disj : requête $\rightarrow$ bool\\
& a-options : requête $\rightarrow$ bool\\
% est-valide : requête $\rightarrow$ bool\\
% note (isabelle) une requête qui est construite est par définition valide (sauf erreur de ma part), donc ce test est inutile.
_accès :& get-word : requête $\rightarrow$ word\\
& get-conj : requête $\rightarrow$ conj\\
& get-disj : requête $\rightarrow$ disj\\
& get-options : requête $\rightarrow$ options\\
\end{tabular}

\paragraph{}
On garde get-conj() si est conj : get-word() puis recherche conjonction et
get-disj() si est disj : get-word() puis recherche disjonction.
\paragraph{}

\begin{tabular}{r l}
\textbf{Type : conj} &\\
_construction :& cons-conj : word * conj $\rightarrow$ conj\\
& cons-conj-simple : word * word $\rightarrow$ conj\\
_test :& est-conj : conj $\rightarrow$ bool\\
_accès :& get-head : conj $\rightarrow$ word\\
& get-tail : conj $\rightarrow$ conj\\
& \\

\textbf{Type : disj}&\\
_construction :& cons-disj : word * disj $\rightarrow$ disj\\
& cons-disj-simple : word * word $\rightarrow$ disj\\
& accès : getDisj()\\
& \\

\textbf{Type : word}&\\ 
_construction :& cons_word : string $\rightarrow$ word\\
_accès :& get-word : word $\rightarrow$ string\\
& \\

\textbf{Type : options}&\\
_construction :& cons-options : word $\rightarrow$ options\\
& cons-options-simple : word * word $\rightarrow$ options\\
_test :& est-options : options $\rightarrow$ bool\\
_accès :& get-head : options $\rightarrow$ word\\
& get-tail : options $\rightarrow$ options
\end{tabular}

