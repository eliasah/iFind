\paragraph{BNF}
Nous avons commencé à réfléchir à la forme qu'aurait la requête de recherche. 
C'est pourquoi nous avons définit une syntaxe et à fortiori, une grammaire sous forme BNF.

Grammaire (BNF) :\\
\begin{tabular}{r c l}
query &$\rightarrow$ & word { -or word } [options]\\
query &$\rightarrow$ & word { word } [options] \\
	 
options&$\rightarrow$ &-e word {word}\\
word   &$\rightarrow$ &char*\\
char   &$\rightarrow$ & [a-z A-Z 0-9 à é è ë ä ù ï ç]\\ 
\end{tabular}

Par la suite, nous avons défini plusieurs types abstraits de données permettant de modéliser la grammaire.
Un type a forcément une ou plusieurs opérations de constructions. 
Chaque type peut présenter des fonctions d'accès et de test ainsi que des constantes.

\paragraph{Type de données abstraites}
En effet, nous avons définit 6 types tels que :

\textbf{Type : operateur}\\
       constantes : plus, OR et Exclu : OPERATEUR
\textbf{requete}\\
\begin{tabular}{r l}
_construction :&  $\rightarrow$ conj $\rightarrow$ requête\\
& $\rightarrow$ conj * options $\rightarrow$ requête\\
& $\rightarrow$ disj $\rightarrow$ requête\\
& $\rightarrow$ disj * options $\rightarrow$ requête\\
& $\rightarrow$ word $\rightarrow$ requête\\
& $\rightarrow$ word * options $\rightarrow$ requête\\ 
& \\
_test :& estValide(requête) $\rightarrow$ bool\\
& estConj(requête) $\rightarrow$ bool\\
& estDisj(requête) $\rightarrow$ bool\\
_accès :& getOptions()\\
	& getWords()\\
\end{tabular}

\paragraph{}
On garde getConj() si est conj : getWords() puis recherche conjonction et
getDisj() si est disj : getWords() puis recherche disjonction.

\paragraph{}

\begin{tabular}{r l}
\textbf{Type : conj} & \\
& cons_conj $\rightarrow$ word * word $\rightarrow$ conj\\
& cons_conj_bis $\rightarrow$ word* conj $\rightarrow$ conj\\
	
&test : estConj(CONJ) $\rightarrow$ bool \\

&accès : getConj()\\

\textbf{Type disj}&\\
&cons_disj $\rightarrow$ word* word $\rightarrow$ disj\\
&cons_disj_bis $\rightarrow$ word * disj $\rightarrow$ disj\\

&accès : getDisj()\\

\textbf{Type : word}&\\ 
&cons_word $\rightarrow$\\ 
&accès : getWords()\\

\textbf{Type : options}&\\
&cons_options : word $\rightarrow$ options\\
&word * options $\rightarrow$ options\\
&test :\\
&estOptions(options) $\rightarrow$ bool\\

&accès : getOptions() \\
\end{tabular}

