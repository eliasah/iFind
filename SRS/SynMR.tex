\subsubsection{Syntaxe du moteur de recherche}

\begin{longtable}{| p{5cm} | p{8cm} |}

\hline
Casse des caractères & iFind ne tient pas compte de la casse des caractères.\\
& \\
& Exemple: Les requêtes \textit{ibm}, \textit{Ibm} et \textit{IBM} renvoient le même résultat.\\


\hline
Lettres accentuées & \textbf{Important} ...\\
& \\
& Exemple: \textit{electricite} et\textit{électricité} ne donnent pas le même résultat, même si les différences
sont souvent minimes.\\

\hline
Orthographe exacte & Il est possible d'utiliser le signe \textbf{+} pour demander une orthographe spécifique. \\
& \\
& Exemple : Une requête contenant \textit{+jéremie} ne traitera pas 
les documents contenant \textit{jeremie} (version non accentuée).\\

\hline
Ordre des mots & \textbf{Important} Une plus grande importance est donnée au premier mot de la requête.\\
& \\
& Exemple: \textit{paris brest} donne un résultat différent de \textit{brest paris}.\\

\hline
Conjonction & \textbf{Opération par défaut}.\\
& La requête est constituée d'une liste de mots séparés par des espaces.\\
& \\
& Exemple : \textit{moteur recherche}\\
& La requête privilégiera les résultats 
contenant \textit{moteur} \textbf{et} \textit{recherche}, 
puis les résultats contenant seulement \textit{moteur},
et enfin les résultats contenant seulement \textit{recherche}.\\

\hline
Disjonction & La requête est constituée d'une liste de mots séparés par la marque \textbf{-or}.\\
& La requête permet de rechercher les résultats contenant un mot seulement parmi la liste,
contrairement à la conjonction.\\
& L'opérateur doit être saisi avec un tiret obligatoirement.\\
& \\
& Exemple : \textit{machin -or bidon}\\
& La requête privilégiera les résultats contenant seulement \textit{machin},
puis les résultats contenant seulement \textit{bidon}.\\

\hline
Exclure un mot & Tout mot ajouté après la marque \textbf{-e} dans la requête 
indiquera que les résultats contenant ce mot sont à exclure.\\
& \\
& Exemple: \textit{moteur –e automobile essence}\\
& La requête donnera uniquemenet les résultats contenant \textit{moteur} mais 
\textbf{ne contenant ni} \textit{automobile}, \textbf{ni} \textbf{essence}.\\

\hline
Type de fichier & Il est possible de préciser quel type de fichier rechercher.\\
& Plusieurs formats sont possibles.\\
& \\
& Exemple: \textit{exemples –f pdf}\\
& Cette requête donnera uniquement les résultats contenant \textit{exemples} et
dont le format de fichier est \textit{pdf}.\\

\hline
Expressions exactes & La recherche d'expressions exactes n'est pas possible.\\

\hline
Troncature & Il n'est pas possible de faire des recherches en utilisant la
troncature sur iFind. Le moteur recherche toujours exactement le mot demandé.\\
& \\
& Exemple: La requête \textit{mot} ne cherche ni \textit{mots} ni \textit{moteur}.\\
& \\
& L'astérique (*) ne peut pas être utilisé.\\
& iFind tient cependant parfois compte de la troncature, sans qu'il soit possible
pour l'utilisateur de décider quand.\\

\hline
Recherche avancée & D'autres options de recherche sont disponibles, telles que la 
recherche par date de mise à jour, par auteur, etc.\\
& Ces critères sont paramétrables via des boîtes de choix au niveau de l’interface graphique.\\
\hline

\end{longtable}
