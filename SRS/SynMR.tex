\section{Syntaxe MR}

\begin{tabular}{l | c}
\hline
Minuscules/ majuscules & iFind ne tient pas compte de la casse des caractères.
Les requêtes 'ibm', 'Ibm' et 'IBM' renvoient le même résultat.\\
\hline
Lettres accentuées & \textbf{Important} \textit{electricite} et \textit{électricité}
ne donnent pas le même résultat, même si les différences sont souvent minimes.\\
\hline
Ordre des mots & \textbf{Important}: \textit{paris brest } donne un résultat 
différent de \textit{brest paris}. Une plus grande importance est donnée au premier 
mot choisi.\\
\hline
Disjonction : Rechercher un mot ou  l'autre, requête -or requête
-or
Exemple : “machin OR bidon”. L'opérateur doit être saisi avec un tiret obligatoirement.
\hline
Conjonction
ET
Opérateur par défaut
Exemple : 'moteur recherche' recherche les fichiers qui contiennent à la fois 'moteur' ET 'recherche'. Il est également possible d'utiliser le signe + pour demander une orthographe spécifique :
Exemple : +jéremie ne trouvera pas la forme 'jeremie (non accentuée)
\hline
 Exclure un mot
-e requête
-e
Exemple : moteur –e automobile recherche les fichiers qui contiennent moteur mais qui ne contiennent pas automobile.
\hline
 Expressions
Non.
Il n'est pas possible de faire des recherches de phrases exactes.
\hline
Troncature
Non
Il n'est pas possible de faire des recherches en utilisant la troncature sur iFind. le moteur recherche toujours exactement le mot demandé. mot ne trouve pas mots ni moteur. L'astérique (*) ne peut pas être utilisé. iFind tient cependant parfois compte de la troncature, sans qu'il soit possible pour l'utilisateur de décider quand.
\hline
 Recherche sur le type   de fichier
-f
Exemple : exemples –f pdf. Plusieurs formats sont possibles.
\hline
 Recherche avancée
Advanced Search, Recherche avancée
Recherche sur le format de fichiers, sur la date de mise à jour, etc. Cependant, il n’y aura pas de syntaxe spécifique mais des boites de choix au niveau de l’interface graphique.


\end{tabular}
