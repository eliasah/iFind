\documentclass[a4paper,10pt]{article}
\usepackage{graphicx}
\usepackage[latin1,utf8]{inputenc}
\usepackage[francais]{babel}
\usepackage[T1]{fontenc}
\usepackage{longtable}
\usepackage{syntax}


%opening
\title{Cahier des charges}
\author{Isabelle Richard, Elias Abou Haydar, \\Mikael Ahl et Jeremie Rouach}

\begin{document}
\maketitle

\tableofcontents
\newpage



\section{Introduction}

\subsection{Objectif du document de specification} 
Ce document de spécification produit des informations spécifiques et nécessaires
pour définir efficacement les fonctionnalités, l'architecture et la conception
du système afin de donner la direction à l'équipe de développement sur
l'architecture du système à développer. Le document de spécification du produit
est créé pendant la phase de planification du projet. Son public visé est le
chef de projet, l'équipe de projet et l'équipe de développement et en partie le
client. Les spécifications techniques et fonctionnelles de ce document sont
réservées au chef de projet, l'équipe de projet et l'équipe de développement.


\subsection{Portée du produit}
Le logiciel iFind permet de rechercher un fichier dans un ensemble de\\
répertoires ciblés du système. Cette recherche peut se faire soit en indiquant
le nom du fichier, soit en donnant une liste de mots contenus dans ce fichier.

% Ajout des définitions, acronymes et abréviations
\subsection{Définitions, acronymes et abréviations}

\begin{description}

\item[UI]
Acronyme de “user interface” (interface utilisateur).

\item[Corpus]
Un corpus est un ensemble de documents, artistiques ou non ( textes, images,
vidéos, etc. ) , regroupés dans une optique précise. 

\item[Fichier]
 Contenant virtuel auquel est assigné un nom unique, permettant de classifier et
de réunir en une même entité une séquence de données. Le fichier est stocké dans
un système de fichier et les données qu'il contient sont généralement
structurées en suivant un même format.

\item[Document]
En informatique, le mot \textit{document} est généralement synonyme de fichier.
On
parle ici de document électronique. Un document électronique est un contenu de
médias électroniques ( autres que les programmes d'ordin\\-ateur ou des fichiers
système ) qui sont destinés à être utilisés soit dans une forme électronique ou
comme sortie imprimée.

\item[Indexation]
L'indexation permet de regrouper en un seul endroit toutes les données
souhaitées. On crée des indexes, ce qui permet d'y accéder plus rapidement. 

\item[Index]
Un index est, en toute généralité, une liste de descripteurs à chacun desquels
est associée une liste des documents et/ou parties de documents auxquels ce
descripteur renvoie. Lors de la recherche d'information d'un usager, le système
accèdera à l'index pour établir une liste de réponses. 

\item[Moteur de recherche]
Un moteur de recherche est un code logiciel qui est conçu pour rechercher des
informations ou retrouver des ressources associées à des mots quelconques. Ici,
on parle de moteur de recherche de type “ Desktop ” , car son champ d'action est
limité à l'ordinateur sur lequel l'application est installée. 

\item[Requête]
 En informatique, une requête est une demande de traitement. Dans notre cas, le
terme est employé dans le contexte des bases de données, une requête
correspondant à l'interrogation d'une base pour en récupérer une certaine partie
des données.
 
\item[Base de données]
 Une base de données, usuellement abrégée en BD ou BDD, est un ensemble
structuré et organisé, permettant le stockage de grandes quantités de
d'informations afin d'en faciliter l'exploration ( ajout, mise à jour, recherche
de données ). Autrement dit, il s’agit d’un conteneur informatique permettant de
stocker dans un même endroit l'intégralité des informations en rapport avec une
activité. Une base de données permet de stocker un ensemble d'informations de
plusieurs natures ainsi que les liens qu'il existe entre les différentes
natures.

\item[Expression régulière]
 Les expressions régulières ( aussi appellées expressions rationnelles ) sont de
chaines de caractères permettant de décrire un ensemble de variables par
l'utilisation d'une syntaxe précise qui se retrouvent dans de nombreux langages
et outils.
 
\item[Démon]
 Un démon ou daemon désigne un type de programme informatique, un processus ou
un ensemble de processus qui s'exécute en arrière-plan plutôt que sous le
contrôle direct d'un utilisateur.

\item[Evènement]
 Un évènement est une action qui a lieu sur un fichier. Les différentes actions 
 possibles sont : création, modification, renommage et suppression.

\end{description}



\subsection{References}
IEEE Std 830-1998 Recommended Practice for Software Requirements Specifications 

\newpage



\section{General Overview and Design Guidelines}
Cette section décrit les principes et les stratégies qui seront utilisées comme
des lignes directrices lors de la conception et de la mise en œuvre du système.

\subsection{Perspective du produit}
iFind utilise une base de données construite à l'aide d'un moteur d'indexation
et mise à jour dès qu'un fichier est modifié. La requête est envoyée au moteur
de recherche via une interface graphique (GUI) (voir Figure 1). On utilise une
interface graphique, pour permettre à l’utilisateur de rechercher ce dont il a
besoin. La GUI est constituée d’un champ de saisie, d’un bouton “Chercher” ainsi
que d’un explorateur qui permettra d’ouvrir les fichiers trouvés (extension). De
base, l’explorateur contiendra un tableau dans lequel on affichera les
résultats. Le champ de saisie reçoit une requête sous forme d’expression
régulière ou des mots simples.

\begin{figure}
\center
\includegraphics[scale=0.7]{rechercheSimple.png}
\caption{Recherche simple}
\end{figure}

Dans cet exemple, la recherche envoie tous les fichiers contenant les mots
"toto" ou "abc".


\subsection{Fonctionnalités du produit}
Lors de la première utilisation, iFind lance un démon ayant pour tâche d’indexer
un corpus ciblé. 
Ce démon va construire une base de données à l’aide de ces index. Un algorithme
est appliqué pour identifier dans le corpus (en utilisant l'index), les fichiers
qui correspondent le mieux aux mots contenus dans la requête, afin de présenter
les résultats des recherches par ordre de pertinence. Cette base de données va
jouer un rôle clé dans la phase de recherche de fichiers.
Ensuite, à chaque création ou modification de fichier appartenant au corpus, le
démon met à jour la base de données en fonction des modifications du fichier.


\subsection{User characteristics}
\subsubsection{Syntaxe du moteur de recherche}

\begin{longtable}{| p{3.5cm} | p{9cm} |}

\hline
Casse des caractères & iFind ne tient pas compte de la casse des caractères.\\
& Exemple: Les requêtes \textit{ibm}, \textit{Ibm} et \textit{IBM} renvoient le même résultat.\\

\hline
Lettres accentuées & Les lettres accentuées sont prises en charge par contre la recherche ne donne pas le même résultat, même si les différences
sont souvent minimes.\\
& \\

\hline
Ordre des mots & On considère une relation d'ordre sur la requête, les mots de la requête sont ordonnés de gauche à droite.\\
& Exemple: \textit{paris brest} donne un résultat différent de \textit{brest paris}.\\

\hline
Conjonction & \textbf{Opération par défaut}.\\
& La requête est constituée d'une liste de mots séparés par des espaces.\\
& \\
& Exemple : \textit{moteur recherche}\\
& La requête privilégiera les résultats 
contenant \textit{moteur} \textbf{et} \textit{recherche}, 
puis les résultats contenant seulement \textit{moteur},
et enfin les résultats contenant seulement \textit{recherche}.\\

\hline
Disjonction & La requête est constituée d'une liste de mots séparés par la marque \textbf{-or}.\\
& Les fichiers trouvés par la recherche contiennent un mot seulement parmi la liste,
contrairement à la conjonction.\\
& L'opérateur doit être saisi avec un tiret obligatoirement.\\
& \\
& Exemple : \textit{machin -or bidon}\\
& La requête privilégiera les résultats contenant seulement \textit{machin},
puis les résultats contenant seulement \textit{bidon}.\\

\hline
Exclure un mot & Tout mot ajouté après la marque \textbf{-e} dans la requête 
indiquera que les résultats contenant ce mot sont à exclure.\\
& \\
& Exemple: \textit{moteur –e automobile essence}\\
& La requête donnera uniquemenet les résultats contenant \textit{moteur} mais 
\textbf{ne contenant ni} \textit{automobile}, \textbf{ni} \textbf{essence}. Remarque l'option -e doît être précédée par un au moins un mots\\

\hline
Expressions exactes & La recherche d'expressions exactes n'est pas possible.\\

\hline
Troncature & Il n'est pas possible de faire des recherches en utilisant la
troncature sur iFind. Le moteur recherche toujours exactement le mot demandé.\\
& L'astérisque (*) ne peut pas être utilisée.\\
& iFind tient cependant parfois compte de la troncature, sans qu'il soit possible
pour l'utilisateur de décider quand.\\
&\\
& Exemple: La requête \textit{mot} ne cherche ni \textit{mots} ni \textit{moteur}.\\
\hline
\end{longtable}


\subsubsection{Grammaire BNF}
\setlength{\grammarparsep}{20pt plus 1pt minus 1pt} % increase separation between rules
\setlength{\grammarindent}{12em} % increase separation between LHS/RHS 

Un mot est formé de chiffres et/ou de lettres minuscules ou majuscules, possiblement accentuées.\\
Une requête doit obligatoirement commencer par un mot.\\
Dans le cas d'une recherche multiple, la liste de mots à rechercher est soit délimitée par 
des espaces (dans le cas d'une conjonction), soit délimitée par le marqueur \textbf{-or} (dans le
cas d'une disjonction).\\
La requête peut être complétée par des options :
\begin{itemize}
 \item la recherche par type de fichiers, avec le marqueur \textbf{-f}
 \item l'exclusion d'une liste de mots, avec le marqueur \textbf{-e}
 \end{itemize}

\paragraph{Grammaire BNF}
\begin{grammar}
<query> ::= <word> [<jonction> ] $[ <options> ]^+$

<jonction> ::= $[ `-or' <word> ]^+$
\alt $[ <wordlist> ]^+$

<options> ::= `-f' <word>
\alt `-e' $<wordlist>^*$

<wordlist> ::= $[ ` ' <word> ]^+$

<word> ::= \textbf{+}$<string>^+$

<string> $\equiv$ $[ a-z A-Z 0-9 ]^+$
\end{grammar}
 
\subsubsection{Utilisation normale}
L'utilisateur entre une requête dans la barre de recherche en suivant la syntaxe 
spécifiée ci-dessus.

\subsubsection{Utilisation avancée}
L'utilisateur peut également entrer une requête incluant des critères spéciaux
sur les fichiers à rechercher :
\begin{itemize}
 \item l'auteur
 \item la date de création
 \item la dernière date de modification
 \item le type
 \item la taille (pour les fichiers de type image)
 \item la durée (pour les fichiers de type musique ou vidéo)
\end{itemize}
Il est possible de paramétrer les fonctionnalités suivantes de l'indexation :
\begin{itemize}
 \item TODO
 \item TODO
\end{itemize}


\subsection{Contraintes générales}
La fraîcheur des résultats dépendra de la fraîcheur de la dernière indexation faite.\\
Le modification du contenu d'un fichier n'est prise en compte qu'au moment de sa
fermeture. Si le contenu d'un fichier est modifié mais qu'une recherche est
lancée alors que ce fichier n'a pas encore été quitté, la recherche ne prendra
pas en compte son nouveau contenu.\\
Un fichier supprimé ne sera jamais retourné comme résultat d'une recherche, la
suppression étant immédiatement traitée.


\subsection{Dépendances}
Le logiciel sera utilisable sur une distribution GNU/Linux.

\newpage



\end{document}
